\chapter{Blockchains are Secure}

{\color{red}
\begin{itemize}
\item The backbone model in the static difficulty setting
\item Interactive Turing machines
\item The environment
\item Synchronous time: The integer round
\item The theoretical network model
\item The rushing adversary
\item The Sybil adversary
\item The parties: The parameters $n$, $t$
\item The honest majority assumption: The parameter $\delta$
\item Mining modeled as a Random Oracle: The parameter $q$
\item The backbone protocol
\item Validating blocks
\item The longest chain rule
\item Mining
\item The Chain Growth property: The parameters $s$ and $\tau$
\item The Common Prefix property: The parameter $k$
\item The Chain Quality property: The parameters $\ell$ and $\mu$
\item Ledger liveness: The parameter $u$
\item Ledger safety
\item Proof of liveness from Chain Growth and Chain Quality
\item Calculation of the liveness parameter $u$
\item Proof of safety from Common Prefix
\item Successful rounds and uniquely successful rounds
\item The probabilistic treatment using the random variables $X$, $Y$, and $Z$
\item Probabilities of success and failure
\item Chernoff bound intuition
\item Chernoff bound theorem for binomial distributions: The parameter $\epsilon$
\item Convergence opportunities
\item Uniquely successful rounds avoid fan-out attacks unless matched by adversarial mining power
\item The equal computational split model
\item The world is a good place: Typical executions
\item The distance between $X$ and $Y$: The block production parameter $f$
\item The distance between $Y$ and $Z$: The Chernoff error parameter $\epsilon$
\item The Chernoff waiting time $\lambda$
\item The balancing equation $3\epsilon + 3f < \delta$
\item Calculating the relationship between adversarial power $(n, t)$, network diameter, and block production rate
\item The Typicality Theorem
\item The Chain Growth Lemma
\item A proof of the Chain Growth property, calculation of the parameters $s=\lambda$ and $\tau=(1 - \epsilon)f$
\item The Chain Slowness Lemma
\item The Pairing Lemma
\item A proof of the Common Prefix property
\end{itemize}
}

\begin{figure}[t]
  \begin{algorithm}[H]
      \caption{\label{alg.backbone} The backbone protocol}
      \begin{algorithmic}[1]
       \Statex
          \Let{\mathcal{G}}{\epsilon}
          \Function{$\textsf{Constructor}$}{$\mathcal{G}'$}
              \Let{\mathcal{G}}{\mathcal{G}'}
              \Let\chain{[\mathcal{G}]}
              \Let{\textsf{round}}{1}
          \EndFunction
          \Function{$\textsf{Execute}$}{$1^\kappa$}
              \Let{\tilde\chain}{\mathsf{maxvalid}_{\mathcal{G},\delta(\cdot)}( \chain, \mbox{any chain $\chain'$ received from the network}) }
  %            \If{$\textsc{Input}()\mbox{ contains }\textsc{Read}$}
  %		        \State{ {\bf write} $R(\chain)$ to \textsc{Output}()}
  %            \EndIf
              \If{$\tilde\chain \neq \chain$}
                  \State\textsc{Broadcast}{$(\chain)$}
              \EndIf
              \Let{x}{\textsc{Input}()}
              \Let{B}{\Call{\sc pow}{x, \tilde\chain}}
              \If{$B \neq \bot$}
                  \Let{C}{C \concat B}
                  \State\textsc{Broadcast}{$(\chain)$}
              \EndIf
              \Let{\textsf{round}}{\textsf{round}+1}
          \EndFunction
          \Function{$\textsf{Read}$}{}
              \Let{x}{\epsilon}
              \For{$B \in C$}
                  \Let{x}{x \concat C.x}
              \EndFor
              \State\Return{$x$}
          \EndFunction
          \vskip8pt
      \end{algorithmic}
  \end{algorithm}
  \end{figure}

  \begin{figure}[t]
  \begin{algorithm}[H]
      \caption{\label{alg.validate} The validate algorithm}
      \begin{algorithmic}[1]
      \Function{$\textsf{validate}_{\mathcal{G},\delta(\cdot)}$}{$C$}
          \If{$C[0] \neq \mathcal{G}$}
              \State\Return$\false$
          \EndIf
          \Let{st}{st_0}\Comment{Genesis state}
          \Let{h}{H(C[0])}
          \Let{st}{\delta^*(st, C[0].x)}
          \For{$B \in C[1{:}]$}
              \Let{(s, x, ctr)}{B}
              \If{$H(B) > T \lor s \neq h$}
                  \State\Return$\false$
              \EndIf
              \Let{st}{\delta^*(st, B.x)}
              \If{$st = \bot$}
                  \State\Return$\false$\Comment{Invalid state transition}
              \EndIf
              \Let{h}{H(B)}
          \EndFor
          \State\Return$\true$
      \EndFunction
      \end{algorithmic}
  \end{algorithm}
  \end{figure}

  \begin{figure}[t]
  \begin{algorithm}[H]
      \caption{\label{alg.maxvalid} The maxvalid algorithm}
      \begin{algorithmic}[1]
      \Function{$\textsf{maxvalid}_{\mathcal{G},\delta(\cdot)}$}{$\overline{C}$}
          \Let{C_\textsf{max}}{[\mathcal{G}]}
          \For{$C \in \overline{C}$}
              \If{$\textsf{validate}_{\mathcal{G},\delta(\cdot)}(C) \land |C| > |C_\textsf{max}|$}
                  \Let{C_\textsf{max}}{C}
              \EndIf
          \EndFor
          \State\Return{$C_\textsf{max}$}
      \EndFunction
      \end{algorithmic}
  \end{algorithm}
  \end{figure}

  \begin{figure}[t]
  \begin{algorithm}[H]
      \caption{\label{alg.pow} The Proof-of-Work discovery algorithm}
      \begin{algorithmic}[1]
        \Function{$\textsf{pow}_{H,T,q}$}{$x, s$}
            \State{$ctr \getsrandomly \{0,1\}^\kappa$}
            \For{$i \gets 1 \text{ to } q$}
                \Let{B}{s \concat x \concat ctr}
                \If{$H(B) \leq T$}
                    \State\Return{$B$}
                \EndIf
                \Let{ctr}{ctr + 1}
            \EndFor
            \State\Return{$\bot$}
        \EndFunction
      \end{algorithmic}
  \end{algorithm}
  \end{figure}

  \begin{figure}[t]
  \begin{algorithm}[H]
      \caption{\label{alg.environment} The environment and network model running
               for a polynomial number of rounds $\textsf{poly}(\kappa)$.}
      \begin{algorithmic}[1]
          \Let{r}{0}
          \Let{\mathcal{T}}{\{\}}
          \Let{Q}{\{\}}
          \Function{$H_{\kappa,i}$}{$x$}
              \If{$x \not\in \mathcal{T}$}
                  \If{$Q = 0$}
                      \State\Return{$\bot$}
                  \EndIf
                  \Let{Q}{Q - 1}
                  \State$\mathcal{T}[x] \getsrandomly \{0, 1\}^\kappa$
              \EndIf
              \State\Return$\mathcal{T}[x]$
          \EndFunction
          \Function{$\mathcal{Z}^{n,t}_{\Pi,\mathcal{A}}$}{$1^\kappa$}
              \State{$\mathcal{G} \getsrandomly \{0, 1\}^\kappa$}
              \Comment{Genesis block}
              \For{$i \gets 1 \text{ to } n - t$}
                  \Comment{Boot honest parties}
                  \Let{P_i}{\textsf{new } \Pi^{H_{\kappa,i}}(\mathcal{G})}
              \EndFor
              \Let{A}{\textsf{new } \mathcal{A}^{H_{\kappa,0}}(\mathcal{G}, n, t)}
              \Comment{Boot adversarial parties}
              \Let{\overline{M}}{[\,]}
              \For{$i \gets 1 \text{ to } n - t$}
                  \Let{\overline{M}[i]}{[\,]}
              \EndFor
              \While{$r < \textsf{poly}(\kappa)$}
                  \Let{r}{r + 1}
                  \Let{M}{\emptyset}
                  \For{$i \gets 1 \text{ to } n - t$}
                      \Let{Q}{q}
                      \Let{M}{M \cup \{P_i.\textsf{execute}(\overline{M}[i])\}}
                      \Comment{Execute honest party $i$ for round $r$}
                  \EndFor
                  \Let{Q}{tq}
                  \Let{\overline{M}}{A.\textsf{execute}(M)}
                  \Comment{Execute rushing adversary for round $r$}
                  \For{$m \in M$}\Comment{Ensure all parties will receive message $m$}
                      \For{$i \gets 1 \text{ to } n - t$}
                          \label{alg.environment.connectivity}
                          \State{$\textsf{assert}(m \in \overline{M}[i])$}\Comment{Non-eclipsing assumption}
                      \EndFor
                  \EndFor
              \EndWhile
          \EndFunction
          \vskip8pt
      \end{algorithmic}
  \end{algorithm}
  \end{figure}



\section{Chain Quality}

\begin{theorem}
  Typical executions satisfy existential chain quality
  with $s = k$.
\end{theorem}
\begin{proof}
  Consider an honestly adopted chain $C$ and a chunk $C' = C[i{:}i + s]$ in it.
  Let $b^*$ be the freshest honestly generated block preceding $C'$,
  and suppose that $b^*$ was computed in round $r^*$ and has height $l^*$.
  Let $b'$ be the oldest block succeeding $C'$
  which an honest party $P'$ has ever attempted to mine on,
  and let $r'$ be the round during which that honest party
  adopted $b'$ and $l'$ be its height.
  Let $S$ be the set of rounds $\{r^* + 1, \ldots, r' - 1\}$.
  By the Patience Lemma, we know that $|S| \geq \lambda$,
  so we can apply typicality to this set of consecutive rounds.
  Consider the convergence opportunities in $S$
  and let $J$ be the set of the block heights of all
  the blocks that correspond to those convergence opportunities.

  These heights $l_j \in J$ will lie in $l^* \leq l_j \leq l'$.
  To see why, note that, after round $r^*$, all the honest parties
  have adopted chains of height at least $l^*$ and so they will
  not mine on anything shorter. Furthermore, if an honest party
  mined a block of height $l' + 1$ or more before round $r'$,
  this block would have been received by $P'$ by round $r'$,
  and so $P'$ would only mine on blocks of length $l' + 1$ or
  more, contradicting the definition of $r'$.

  Consider which block $B$ among those that arose in convergence
  opportunities during $S$ form part of $C'$. If the block $B$
  with height $l$
  is \emph{not} a part of $C'$, then this means that the
  honest party has adopted a different block $B'$ of height $l$ in
  its stead, as $C'$ has height at least $l$. By the Pairing Lemma,
  $B'$ must be adversarially produced. In the same fashion, we can
  pair every height in $J$ that does \emph{not} form a part of $C'$
  with a separate adversarially successful query.

  These adversarial queries all lie in $S$. To see why, note that
  they must be after $r^*$ due to causality.
  {\color{red} Something must be said about the adversarial query
  being before $r'$.}

  Therefore, $Y(S) - Z(S)$ honestly generated blocks must be in
  $C'$. By typicality $Y(S) > Z(S)$, and so existential quality
  holds.
  {\color{red} We need to still argue that $Y(S) - Z(S) \geq 1$.}
\end{proof}

Based on the above proof, we can calculate the exact chain quality,
too.
The number of surviving honest blocks in $S$ rounds is
$Y(S) - Z(S) \geq (1 - \frac{\delta}{3})f|S| - (1 - \frac{2\delta}{3})f|S|
= (\frac{2\delta}{3} - \frac{\delta}{3})f|S| = \frac{\delta}{3}f|S|$.
During these rounds, the total number of blocks by which the chain has
grown must be at least $(1 - \epsilon)f|S|$ by chain quality. Therefore,
the proportion of surviving honest blocks is
$\mu = \frac{\frac{\delta}{3}f|S|}{(1 - \epsilon)f|S|} = \frac{\delta}{3(1 - \epsilon)}$.

\section*{Problems}

\begin{enumerate}
  \item Consider an execution under a minority adversary and good parameters such that
        the Common Prefix property is typically preserved for some parameter $k$. Now,
        under the same parameters, you, the adversary, are given a special power: You
        can change the response to exactly \emph{one} fresh random oracle query by the
        honest parties. You can change the query adaptively (you can decide whether you
        want to change it, or not, once you see the query). You cannot change an old query
        for which the random oracle already has a cached entry. After you change one response,
        the random oracle caches your response and responds consistently in the future.
        How would you use this ability to break Common Prefix for the same parameter $k$?
\end{enumerate}
