\chapter{Light Clients}\label{chapter.light}

So far, all the participants in our network have been downloading the whole blockchain,
including all transactions, and rebuilding the transaction graph to verify that everything
is well. This includes both miners, as well as wallets, who need to have access to a node.
A node that performs this whole verification is known as a \emph{full node}\index{Full Node}.

As the blockchain contains all transactions ever made by anyone, this can be quite inefficient
if the purpose is to just operate a wallet. After all,
why does a wallet, which concerns itself about only very specific and limited public keys,
need to know about everyone's transactions? It is only interested in transactions that concern
the public keys it holds, as well as verifying that any incoming money is truly existent and
not forged.

As time goes by and the blockchain grows, this problem becomes more pronounced, because transactions
are only ever added but never removed. The larger the running time of the system, the more data
that has to be downloaded by a node, and the more time this takes.
This is especially bothersome for a node that holds only the genesis
block and is booting into the system for the first time. This problem is known as the
\emph{bootstrapping problem}\index{Bootstrapping Problem}.
From a practical standpoint, a blockchain system is unusable if a mobile wallet app takes
a few days to synchronize with the rest of the network when first installed, and in the meantime
uses all of the user's bandwidth to download this data and a significant amount of battery
for signature validation.

Of course, a mobile app could use a custodian wallet that holds its private keys, or
utilize a trusted node while holding its own private key locally. However, this entails
a level of centralization that we would like to avoid.

Towards this end, we will modify our blockchain system to allow a node to
synchronize quickly. Our nodes will only download transactions that are of interest
to them, as well as the proof-of-work of the chain, but no transactions that do not
concern them.
These nodes are known as \emph{light nodes}\index{Light Node}, and the protocol
we will develop is called \emph{simple payment verification}\index{Simple Payment Verification} (SPV).

{\color{red}
\begin{itemize}
\item The Random Oracle model
\item Random Oracles are collision resistant
\item Random Oracle collision resistance security proof
\item Random Oracles are pre-image resistant
\item Random Oracle pre-image resistance security proof
\item Block predictions are difficult in the Random Oracle model
\item Authenticated data structures
\item Merkle trees
\item Building a complete Merkle tree from data
\item Building a Merkle tree proof
\item Validating a Merkle tree proof
\item Merkle trees in the torrent protocol
\item Merkle tree game-based security definition
\item Merkle tree security theorem and proof
\item The SPV protocol
\item Light clients
\item Sparse Merkle trees, Merkle-Patricia tries: Polynomial data in an exponential sea
\item Light mining using block state commitments
\item State commitment in block headers
\end{itemize}
}
