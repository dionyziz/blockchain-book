\chapter{The Application Layer: UTXO}

{\color{red}
\begin{itemize}
\item 
  Goal: Transparency / verifiability (everyone knows that the money they claim they have, they *actually* have)
\item
  What is money? How do we know today who owns what? Did we track it through history? Do we have that history? No -- we trust our social environment. A kind of "honest majority" setting.
\item The adversary
  \begin{itemize}
    \item Problems hard in the average case VS problems hard in the worst case
  \end{itemize}
\item Commitment schemes
\item The binding and hiding games
\item Hashes
\item The collision resistance and pre-image resistance games
\item Public and private keys
\item Signatures, signing, verifying, correctness, security
\item The existential unforgeability game
\item A transaction
\item Addresses
\item One input and one output
\item Amounts
\item Inputs and outputs in UTXO
\item Splitting money
\item Merging money
\item Conservation Law
\item Change
\item Outpoints -- the transaction graph
\item The transaction ledger / transaction serialization
\item UTXO as application state
\item The evolution of the UTXO
\item What money do I own? Calculating balances
\item Double spending
\item Reading: Chapter 5 introduction and section 5.1 from Modern Cryptography (2nd ed.)
\item Reading: Chapter 12 introduction (skip comparison to MACs and relation to encryption) and sections 12.1, 12.2, 12.3 from Modern Cryptography (2nd ed.)
\item Exercises 2 and 3
\end{itemize}
}
